\documentclass[12pt]{article}

\title{Task Free Recall}

\author{Computational Memory Laboratory\\University of Pennsylvania}

\begin{document}

\maketitle

\section{Procedure}
38 participants from the University of Pennsylvania
community received payment in accordance with
the University's IRB guidelines. Stimuli were presented with
a computer running PyEPL (Python Experiment Programming
Library: http://pyepl.sourceforge.net, Geller,
Schleifer, Sederberg, Jacobs, \& Kahana, 2007). Verbal responses
were recorded with a microphone and parsed with
the pyParse package.

Participants ran in 4 sessions.  Each session consisted of 12 free recall lists, a final free recall period, and a recognition period.

\subsection{Stimuli}
Words were chosen from a subset of the Toronto Noun Pool.  All the words used have associated Word Association Space (WAS) scores.  Words were excluded if they were ambiguous in meaning or were ill-defined with respect to the size and animacy judgments used in the experiment.  For example, ``METAL'' was excluded since its size is ill-defined.  The final word pool contained 1655 words.

\subsection{Free Recall}
On each trial, a list of 24 words was presented; each word
was presented with a task cue above it, indicating the
judgment that the participant should make for that word (``Size'' or ``Living/Nonliving'').  
Each word was presented for 3 seconds. The two tasks were
a size judgment (�Will this item fit into a shoebox?�) and
an animacy judgment (�Does this word refer to something
living or non-living?�).  Participants made their response to each word using the index and middle fingers of their right hand to press one of four keys labeled ``Big'', ``Small'', ``Living'', and ``Nonliving''.  If participants failed to respond during the time the item was onscreen, a beep was presented along with a message asking them to respond more quickly.  If participants pressed one of the two keys that was inappropriate for the current task (e.g. ``Living'' if asked to make a size judgment), a beep was presented with a message asking them to press one of the appropriate keys.  Half of each participant's performance-based bonus pay (\$5 per session) was based on how many slow or inappropriate responses they made (the other half was based on the number of times they blinked during word presentations, as measured by EOG recordings).

The screen was blank for a 1000$\pm$200ms inter-stimulus interval (ISI) between each word.  ISI was randomly varied in order to de-correlate brain activity associated with each word presentation from activity during subsequent word presentations. Immediately following the list, a row
of asterisks appeared, along with a beep, indicating the start
of the recall period. Participants were given 90 seconds to
recall as many words as they could remember from the most
recent list, in any order.

There were two trial conditions, control and task shift. On
control lists, every word was judged with the same encoding
task. On the task-shift lists, participants shifted back
and forth between the two tasks. Words were presented in groups, or ``trains'', where a number of contiguous items were studied with the same encoding task. The length of each train was random, with the constraints that each train was between
2 and 6 items long (inclusive), and that the total number of items studied with each task was equal within the list.  We counterbalanced (across lists) the task
used to start the list and the number of trains in the list (6 or 7).  This was done to balance whether lists started and ended with the same task, or started and ended with different tasks.  The number of control and shift lists was balanced within each session.  Lists were ordered in groups of four, where each group contained two control (one of each task) and two shift lists, and the order of lists was randomized within each group.

Using pilot data where participants were asked to rate each word using both encoding tasks, the words on a given list were chosen such that in total
there would be a roughly equivalent number of items associated
with each response (�big�, �small�, �living�, and
�nonliving�). Many words are ambiguous with regard to the
�correct� judgment (e.g., given the word �dog�, an image of
a chihuahua might elicit a �small� judgment, while an image
of a Great Dane might elicit a �big� judgment).  Items for each list were chosen so that no two items were above a certain threshold of WAS similarity.

\subsection{Final Free Recall}
Once all the free recall trials were completed, participants were asked to recall words from the entire session, in any order.  Participants were given six minutes to recall as many words as they could.

\subsection{Recognition}
Following the final free recall period, participants were presented with words that either had been presented during the session (target) or had not been presented during the session (lure).  Complete lists from the session were randomly chosen for inclusion in the pool of targets. (NWM: for taskFR2, this was changed to choosing individual words at random.  The choice to grab complete lists was deliberate, but no one seems to remember why we did it.)

\begin{table}
	\centering
	\begin{tabular}{|c|c|c|}
		\hline
		\textbf{Ratio}	&	\textbf{\# Lures}		&	\textbf{\# Targets} \\ \hline
		0.125		&	41				&	288                        \\ \hline
		0.250		&	96				&	288                        \\ \hline
		0.375		&	158				&	264                        \\ \hline
		0.500		&	192				&	192                        \\ \hline
	\end{tabular}
	\caption{The four conditions for the recognition period.  ``Ratio'' indicates number of lures divided by total number of stimuli.}
	\label{tab:recog_conditions}
\end{table}

The (Number of lures)/(Number of Targets and Lures) ratio was varied across sessions within each participant.  The four conditions are reported in Table \ref{tab:recog_conditions}.  The order of conditions was counterbalanced across participants according to a latin square.

Order of presentation of items was randomized.  For each word that was presented, participants were asked to verbally indicate whether the item was a target.  Participants were instructed to answer by saying either ``pess'' for yes, or ``po'' for no.  This was done to facilitate subsequent scoring of response onset times.  

After responding yes or no, participants rated their confidence in their response on a scale from 1 (not at all sure) to 5 (completely confident).  Participants were instructed to make their yes or no response as soon as they were sure which way they were leaning, then take extra time to decide their confidence if necessary.  The first ten participants (LTP001-LTP010) made their confidence rating verbally; after speaking their response and confidence rating, they pressed a key to move on to the next word.  Participants LTP011-LTP038 made their confidence rating by pressing 1-5 on a numeric keypad, which caused the program to advance to the next word.  

The screen was blank for a 1000$\pm$200ms inter-stimulus interval (ISI) between each word.  ISI was randomly varied in order to de-correlate brain activity associated with each trial from activity during subsequent trials.  Participants were given a chance to rest after each block of 20 items.

\section{EEG Recordings}
EEG measurements were recorded using commercially available equipment (Net Amps 200 amplifier, Net Station 4.1 acquisition software, Electrical Geodesics, Inc.).  Recordings were made from 129 electrodes, and digitized at a sampling rate of 500 Hz.  Recordings were originally referenced to Cz, and later converted to an average reference.  Electrodes that had poor contact with the scalp were identified through manual inspection of raw EEG traces, and were excluded from the re-referencing.

In order to identify epochs contaminated with eye motion artifacts, recordings were made from electrooculogram (EOG) channels.  Epochs were excluded from analyses if the weighted running average for either the left or right EOG pair exceeded a 100 $\mu$V threshold.

\end{document}